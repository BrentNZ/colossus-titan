\documentclass{article}
\usepackage[latin1]{inputenc}
\usepackage{html}

\begin{document}

% $Header$

\title{The Colossus Battlelands Builder}

\author{Romain Dolbeau}

\maketitle

\section{BattlelandsBuilder}

BattlelandsBuilder is a (crude) tool to interactively
create Battlelands for use with the
``\htmladdnormallink{Colossus}{http://colossus.sf.net/}'' Game.
It has the same requirements as Colossus itself, see
\htmladdnormallink{Colossus README}{../../README/index.html}
and
\htmladdnormallink{Colossus build-HOWTO}{../../build/index.html}.

To create the program, just type ``make tools'' or ``and tools''.
See the
\htmladdnormallink{Colossus build-HOWTO}{../../build/index.html}
for more details.
This should create all the tools for Colossus, including
BattlelandsBuilder.

You can use it directly, starting from a bare plain,
or you can give a pre-existing Battlelands as a
starting point. Just give the path to the file as
the first argument to the command.

Exemples:
\texttt{java -jar BattlelandsBuilder.jar}
\texttt{java -jar BattlelandsBuilder.jar Battlelands/Woods}

Editing is point-and-click : click in the middle
of an hex to get a popup-menu to change the type
and elevation of the hex ; click near the border of
an hex to get a popup-menu to change this border of
this hex (warning: beware slope or wall who can
easily be put on the wrong hex, as there's not much
visual clue of their orientation).

If at least on hex is selected (surrounded by red),
that means the terrain use a startlist, i.e. custom
entry hexes for the defender (like the Tower in regular
Titan). If no hexes are selected, normal rules apply.

The ``File'' Menu:

\begin{itemize}

\item The ``Load Map'' menu item is used to load a Battlelands 
file in the map. The loaded battleland is \emph{merged}
with the currently displayed battleland. Use ``Erase Map''
below if you want to start from scratch.

\item The ``Save As'' menu item save the Battleland to a
file choosen by the user.

\item The ``Show Battleland'' menu item dump the Battleland
on standard output (can be seen only if you have a CLI
or a way to access standard output...)

\item The ``Erase Map'' menu item return the map to the
default, i.e. only Plains of elevation 0, and no
hexside. Warning: startlist is unchanged.

\item The ``Randomize Map'' menu item is used to generate
a random map from a description file. See the
\htmladdnormallink{BattlelandsRandomizer}{../BattlelandsRandomizer/index.html}
for a description of the input file.

\end{itemize}

The ``Special'' Menu:

\begin{itemize}

\item The ``Terrain is a Tower'' is used to specify if the terrain is designed to be a Tower or not (i.e. the \texttt{TOWER} flags explained in \htmladdnormallink{FileFormat}{../../FileFormat/index.html}).

\item The ``Remove StartList'' remove unselect all hexes. See the file format
link above for a startlist description.

\end{itemize}

\section{Frequently Asked Questions}

\begin{enumerate}

\item[Q] I can't get the Hex Side popup-menu !
\item[Q] My Hex Side is on the wrong hex !

\item[A] To get the Hex Side menu, you must be quite close
to the Side itself. You must also not be on one
of the outer side, without a neighbor Hex, as you
can't put an Hex Side there. Finally, if the Hex
Side is put ``backward'', that means you clicked on
the wrong Hex ; for instance, you must click behind
the straight border, on the Sand, to get a proper
Dune. Clicking where you intend to get the round
part will put the Dune backward.

\item[Q] BattlelandsBuilder let me do something dumb!

\item[A] Well, yes. It's designed that way : you can do
whatever you please with it, including surrounding
a Tree with dune or filling the map with Volcanoes.
It will try to flag some obviously ``dangerous''
or ``wrong'' combinations with warnings, but you
can use the output anyway (the warnings are
comments that will be ignored by Colossus).

\end{enumerate}

\end{document}
