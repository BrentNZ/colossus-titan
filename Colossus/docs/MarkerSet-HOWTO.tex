\documentclass{article}
\usepackage[latin1]{inputenc}
\usepackage{html}

\begin{document}

%$Header$

\title{HOWTO Add A Marker Set to Colossus}

\author{Romain Dolbeau}

\maketitle

\section{Introduction}

This document explains how to add a set of stack markers to Colossus.

\section{Requirements}

To add a new set you'll need to supply:

\begin{itemize}

\item A new Color, preferably one that cannot be confused with any other, and that cannot be confused with the Masterboard Hexes.

\item A two letters identificator for the Color, that cannot be confused with any other (see the file ``net/sf/colossus/server/Constants.java'', the two variables \texttt{colorNames} and \texttt{shortColorNames}). It's better if it's meaningful, as it is displayed to the player(s).

\item A single letter color mnemonic used for Color selection. Must be unique. See again ``net/sf/colossus/server/Constants.java'', array \texttt{colorMnemonics}.

\item Twelve 56x56 GIFs file with a unique name each (see ``Default/MarkersName''). Names must be the 2-letter description followed by the number (between 01 and 12), with the \texttt{.gif} extension. They probably can be PNG files with a \texttt{.png} extension.

\end{itemize}

\section{Adding the set}

\begin{enumerate}

\item Add the color to ``net/sf/colossus/util/HTMLColor.java'' as a static member. The name must be the full color name with a ``Colossus'' suffix.

\item Add the color name, short name and mnemonic to ``net/sf/colossus/server/Constants.java'', in the three relevant arrays (\texttt{colorNames}, \texttt{shortColorNames} and \texttt{colorMnemonics}). Note that the order is important : AI will pick up the first 6 colors before all others, and the order must be the same in all three arrays.

\item Add the image files to the ``Default/images/'' directory.

\item Add the names in ``Default/MarkersName'' ; each name must be associated to the corresponding short description (2-letter + number).

\item Increment the \texttt{MAX\_MAX\_PLAYERS} constants in ``net/sf/colossus/server/Constants.java''.

\end{enumerate}

The Titan Chit is colored on the fly using the color specified in the ``HTMLColor'' file, so you don't have to worry about it.

\end{document}
