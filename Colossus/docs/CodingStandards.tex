\documentclass{article}
\usepackage[latin1]{inputenc}

\begin{document}

% $Header$

\title{Coding standards for Colossus}

\author{David Ripton}

\maketitle

\begin{enumerate}

\item Do not put tab characters in code, period. Use spaces only.
There is no standard tab stop in Java, so tabs will look bad for
anyone whose editor is set up differently than that of the person
who put tabs in the code. Configure your editor to emit the
appropriate number of spaces rather than a hard tab when you hit
the tab key. (Of course, Makefiles are an exception, because
you have to use tabs in them. Ick.) ``ant fix'' converts tabs
to spaces for you, but it assumes a tab stop of 8 unless you
change it.

\item Wrap code at 79 characters. This is a pain sometimes, but it
allows easly working with the code using 80-column editor and
terminal windows. (Remember, code is not just viewed through
your favorite editor in your favorite GUI desktop. It's also
emailed around, diffed, viewed in debugger windows, fiddled with
on remove servers through network connections, etc.)

\item Put opening braces at the beginning of a new line rather than
the end of the previous line. This decision has been argued for
decades and will never be settled (at least not until all language
designers follow Python's lead and just ditch the braces), but
consistency within a project is more important than personal
preference.

\item Colossus currently compiles and runs under JDK 1.2. Try to
keep it that way. At some point we might decide to bump the JDK
version requirement for a truly great feature (e.g. templated
container classes), but don't do it trivially. 

\item Keep all member variables private, except for constants.
Use get/set methods to share. (But don't automatically add 
an accessor and mutator for every variable, just for the ones 
that actually need them.) This adds some overhead up front 
but makes it easier to refactor code later.

\item Try to comment stuff, at least at the method level for
non-trivial and non-private methods. But there's no need to go 
overboard commenting obvious stuff. Use javadoc comments
instead of regular comments where applicable.

\item Every method in the .client and .server packages should be
either private (if possible) or package private (if needed
by other classes in the package), unless it needs to be public.
And just because a method is public doesn't mean it should be
called from outside the package. Remember, there will only
be a network interface between Client and Server.
\label{permission}

\item Some methods are private because they're not used outside of
their class, not necessarily because they shouldn't be used
outside of their class. Some classes are final because they're
not currently subclassed, not because they should never be
subclassed. These are optimizations for both programmers (who
don't need to check to see if other classes call / override
something before changing it if it's tagged private / final) and
for the JVM, not necessarily design statements. So feel free
to change such things as necessary, until the interfaces are
finalized and polished and other people depend upon them. But
see \#\ref{permission}.

\item Always do a cvs diff before you check code in. That way
if your editor accidentally changed a bunch of spaces to tabs,
or changed every \texttt{CR} to a \texttt{CR/LF}, or the broken
change you thought you took out somehow got back in, you'll know.
And you'll also be able to write a better check-in comment because
you won't forget about that other change you made yesterday.

\item Fully brace conditionals, even if the clause is only one line
and doesn't technically need braces. This makes the code taller,
but easier to read and change.

\item Keep files in Unix endline format (\texttt{LF}), not Windows
format (\texttt{CR/LF}) or Mac format (\texttt{CR}). This is critical
for Makefiles, which will sometimes stop working if the line breaks
are wrong. Java files will work either way, but it's annoying to have a
mix of line breaks, false diffs caused by files getting converted back
and forth, etc. so consistent Unix format is best. ``ant fix'' is your
friend.

\item Otherwise just try to follow the existing format. A foolish
consistency may be the hobgoblin of little minds, but it sure
makes code easier to read.

\end{enumerate}

Thanks for reading this.

\end{document}
