\documentclass{article}
\usepackage[latin1]{inputenc}
\usepackage{html}

\begin{document}

% $Header$

\title{Colossus Variant HOWTO}

\author{Romain Dolbeau}

\maketitle

\section{The Variant Pack}

\subsection{How do I use a Variant Pack ?}

Just use the ``Load External Variant'' button and open the
\texttt{VAR} file inside the variant directory, then start a new
game. That should be enough.

Alternatively, some variants are now supplied in the
standard distribution ; simply use the pop-up menu
to select which variant to play (``Default'' is the
Classic Titan ruleset).

\subsection{What's in a Variant Pack ?}

A Variant Pack should be a directory (Folder for
MacOS user) containing:

\begin{itemize}
\item a \texttt{VAR} file, that describe the other files;
\item a \texttt{CRE}, \texttt{TER} and \texttt{MAP} files that describe the variant;
\item a ``Battlelands'' subdirectory that contain the new and/or changed Battlelands;
\item an ``images'' subidirectory that contain the new and/or changed pictures.
\end{itemize}

The various files are described in details inside the \htmladdnormallink{Files Formats}{../FileFormat/index.html} document.

\subsection{How do I create a Variant Pack ?}

Well, there's no ``user-friendly'' way yet. The easiest way is
to build the directory and copy the default files in it,
then you can change them with any text editor. The contents
of the files are described in
\htmladdnormallink{Files Formats}{../FileFormat/index.html}.
You can also use other Variants as examples.

Warning: almost no sanity check is performed, and some
combinations might give strange and/or unexpected result,
particurly in Battlelands (slope next to a tree, or
cliff between two bogs...). Also, ensuring consistency
between the files is up to the Variant author.

Battlelands can also be created using the graphical tool
``\htmladdnormallink{BattlelandsBuilder}{../datatools/BattlelandsBuilder/index.html}''

There may be an easier (read: graphical) way later for
\texttt{CRE}, \texttt{TER} and \texttt{MAP} files, but don't hold your breath, nobody
is working on it (yet).

\section{Frequently Asked Questions}

\begin{enumerate}

\item My creature won't load !
\item My creature is displayed as a dull grey square !

Creature's name (the first column of a \texttt{CRE} file) should not
contain space or other non-alphabetical character (you
can't use digits yet), unless put between doublequotes ``"''.
i.e., proper name are ``KillerRabbitTwo'' or ``"Killer Rabbit 2"''.
The same name must be used in the \texttt{CRE} and \texttt{TER} file.

The creature's name is also used as the base name
of the pictures. So a ``KillerRabbitTwo'' will use
the file ``KillerRabbitTwo.gif'' in the subdirectory
``images''. Spaces will be replaced by underscores ``\_'',
and doublequotes will be ignored. So ``"Killer Rabbit 2"''
will use ``Killer\_Rabbit\_2.gif''.

You should be able to re-use an existing name, as
the ``images'' subdirectory is accessed before the default
\texttt{GIF} files.

\end{enumerate}

\end{document}
