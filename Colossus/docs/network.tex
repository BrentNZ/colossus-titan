\documentclass{article}
\usepackage[latin1]{inputenc}
\usepackage{html}

\begin{document}

% $Header$

\title{Colossus network play}

\author{David Ripton}

\maketitle

\section{Networked Colossus}

Colossus is now playable over a TCP/IP network.  However, the network 
functionality is still new and probably buggy.  And turn-based games
are inherently brittle -- if one player stops moving or drops his 
connection, then the game is ruined for everybody.  (For now -- we'll
add the ability to reconnect or have an AI take over later.)  So I
recommend starting with 2-player games rather than 6-player games.

One person needs to run the server.  For now, Colossus servers are 
short-lived, staying up only for the length of one game.  The server
needs a local copy of the Colossus code, and a Java Development Kit 
or Java Runtime Environment.  The JDK/JRE bin directory needs to be 
in the PATH.  And the server computer needs to have a well-known 
hostname or IP address accessible to the other players' computers, 
without draconian firewalling.

The server operator needs to run the runserver script, which starts
the rmiregistry program (if it's already running, this will fail and 
throw an exception, which won't hurt anything) and then sets RMI 
codebase and security policy locations and runs the server.  Because
the codebase is set relative to the current directory, you need to run
this script from its own directory, the base of the Colossus source tree.

(Note: if you use the run script instead of the runserver script,
remote clients won't be able to connect.)

rmiregistry is a program included with the JDK/JRE that sets up named 
RMI connections.  It binds to port 1099 by default.

The startup dialog should appear, allowing you to set the variant,
optional rules, and how many players of which types you want.  Human
players are local to the server machine.  Network players are remote.
The game will wait for *exactly* the number of network players you 
specify, then start the game.  If you prefer, you can use command-line 
options instead.  ``\texttt{run -h}'' to see them.

Each client needs to wait for the server to be started, then run
the runclient script.  It takes two arguments.  -p playername and
-s server host name.  The player name will default to your OS user
name if you don't specify one.  The server host will default to the
local machine, which probably isn't what you want unless you're doing
loopback testing.  So you'll want to run it like
``\texttt{runclient -s hostname}''
You can use the IP address instead of the hostname if you prefer.

Once all the clients have connected, then the game should start up like
a normal Colossus game.  When the game ends, kill all the clients and
the server and restart from scratch next time.

We don't have a built-in chat client yet, so you might want to run
your favorite IRC or instant messenger client in the background. 


\section{FAQ}

\begin{itemize}

\item[Q] I can't make it work because of my firewall.

\item[A] That means your firewall is working as designed.

   If you control the firewall, configure it to allow inbound and 
   outbound traffic on all ports > 1024 while playing.  If you don't 
   control the firewall, you're probably hosed -- the game currently
   uses Java RMI, which is not at all firewall-friendly.
   
   I plan to ditch RMI in favor of simple sockets with server-to-client 
   traffic piggybacking on the original client-to-server connection.

\end{itemize}

\end{document}
