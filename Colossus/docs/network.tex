\documentclass{article}
\usepackage[latin1]{inputenc}
\usepackage{html}

\begin{document}

% $Header$

\title{Colossus network play}

\author{David Ripton}

\maketitle

\section{Networked Colossus}

Colossus is now playable over a TCP/IP network.  However, the network 
functionality is still new and probably buggy.  And turn-based games
are inherently brittle -- if one player stops moving or drops his 
connection, then the game is ruined for everybody.  (For now -- we'll
add the ability to reconnect or have an AI take over later.)  So I
recommend starting with 2-player games rather than 6-player games.

One person needs to run the server.  For now, Colossus servers are 
short-lived, staying up only for the length of one game.

The server operator simply starts the game as usual, using Java Web
Start or the ``run'' script for a local copy.  The startup dialog 
should appear, allowing you to set the variant, optional rules, and 
how many players of which types you want.  Human players are local 
to the server machine.  Network players are remote.  The game will 
wait for *exactly* the number of network players you specify, then 
start the game.  If you have a local copy of the game you can use 
command-line options instead.  ``\texttt{run -h}'' to see them.

Each client needs to wait for the server to be started, then run
the runclient script or click on the client Java Web Start icon.  
It will pop up a dialog that lets you enter a player name, hostname, 
and port.  The player name will default to your OS user name.  The 
server host will default to the local machine, which probably isn't 
what you want unless you're doing loopback testing.  You can use the 
IP address instead of the hostname if you prefer.  Most servers will 
use the default port.  Again, you can use command-line options 
instead.  ``\texttt{runclient -h}'' to see them.

Once all the clients have connected, then the game should start up like
a normal Colossus game.  When the game ends, kill all the clients and
the server and restart from scratch next time.

There is a simple chat client built in, but it doesn't persist across
multiple games, so you might want to run your favorite IM or IRC client
alongside the game.

\section{FAQ}

\begin{itemize}

\item[Q] What about firewalls?

\item[A] We changed from RMI to simple sockets, so the game should be
much more firewall-friendly.  The server accepts connections from 
clients on one port, and piggybacks return traffic on the same sockets
rather than opening more.  The server also use a second port to serve
data files to clients. So as long as the client machines have the
ability to connect out to high ports, and the server machine has two
consecutive inbound ports open, it should work.

Default ports are 26567 and 26568, but they can be changed at game
startup time (the port must be consecutive ATM, you give only the
first).

\end{itemize}

\end{document}
