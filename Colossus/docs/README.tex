\documentclass{article}
\usepackage[latin1]{inputenc}
\usepackage{html}

\begin{document}

% $Header$

\title{Colossus README}

\author{David Ripton}

\maketitle

\section*{Introduction}

Colossus is an attempt at a Java clone of Avalon Hill's Titan(tm) boardgame.

It's not finished yet. Right now it allows hotseat play, and play against a 
mostly working but not-quite-ready-for-prime-time AI. Client/server
networking is done but still a bit rough. You can play with the standard Titan
rules, or choose from several variants.

This program is freeware, distributed under the GNU public license, which is 
included in the file COPYING.GPL. This means that you have the right to make 
and distribute changes, as long as you always include the source code so that 
others can do the same. If you fix any bugs or add any features, please send 
us a copy so that we can fold them into the master code.

\section{Game requirements}

A 1.3 or later version of a JRE (Java runtime environment) or JDK (Java 
development kit). 

Unless you're an advanced Java user, then you should probably use Java Web 
Start, which will automatically download and install the correct JRE for you.
\htmladdnormallink{http://java.sun.com/products/javawebstart/}{http://java.sun.com/products/javawebstart/}

(Colossus will not run under JRE 1.0.x or 1.1.x or 1.2.x  It won't run as an 
applet in a web browser. The obsolete Microsoft JVM that's bundled with IE and 
Windows won't work.)

Windows, Solaris, and x86 Linux versions of the JRE and JDK are freely 
downloadable from \htmladdnormallink{http://java.sun.com}{http://java.sun.com}.

Another Linux port is available from
\htmladdnormallink{http://www.blackdown.org}{http://www.blackdown.org}.

The Mac version is at
\htmladdnormallink{http://devworld.apple.com/java/}{http://devworld.apple.com/java/}.

Info on other ports (AIX, OS/2, etc.) is at 
\htmladdnormallink{http://java.sun.com/cgi-bin/java-ports.cgi}{http://java.sun.com/cgi-bin/java-ports.cgi}.

The current recommended version is Sun JRE 1.4.0 on platforms that have it,
(Windows, x86 Linux, Solaris) and 1.3.1 on platforms that don't.  (We know of 
some GUI bugs in 1.3.1 that are fixed in 1.4.0.)

You also need a computer with a mouse and a color display.  The minimum system
spec is about a Pentium 133 with 64MB.  The recommended system is about a 300 
MHz CPU and 128MB.  It's free, so if you don't know if your system is fast 
enough, just try it.

\section{Getting the program to run}

The easiest way to run the game, if you already have Java Web Start (or an 
equivalent JNLP launcher) installed on your computer, is to click on the Java 
Web Start link on the web page 
(\htmladdnormallink{http://colossus.sf.net}{http://colossus.sf.net}).  
This will download the latest version of the game, upgrade your JRE if 
necessary, make a shortcut icon if you want, etc.  Really slick, when it works.
If it doesn't work, you have other choices.

Another option is to run the executable jar file. If you downloaded 
Colossus.jar by itself, you're set. If you downloaded a zip file, you will need
to unzip it using your favorite unzip tool and find the jar file inside. Try 
double-clicking on Colossus.jar in your GUI file manager.  If that fails,
pop up a command prompt, cd to the directory where you unzipped the zip file, 
and try typing ``java -jar Colossus.jar''

Yet another option, in case the jar file is temporarily broken for some reason, 
is to expand the whole game tree and run the Start class manually.  Unzip the 
zip file and try ``java net.sf.colossus.server.Start''  (Or use ``run'' or
``run.bat''.)

\section{Game play instruction}

I assume you already know how to play Titan. The full rules are copyrighted by 
Avalon Hill, so I can't provide them. Titan is an excellent boardgame, and I 
recommend that everyone buy a copy while you still can. (Avalon Hill was bought 
by Hasbro, which may or may not decide to reprint Titan in something like its 
original form.)

Once you get things running, a dialog should pop up, allowing you to choose up 
to six player types (human, AI, or not present) and their names.  You can also 
choose a variant.  When you're done, click ``OK'' 

Another dialog will pop up, telling you which tower each player gets and 
letting players pick colors in increasing order of tower number. Pick colors 
for each human player.

Now a window will pop up for each player, letting him pick his initial legion
marker. Pick one. (If you really don't care, use the ``Auto pick markers''
option on the Player menu.)\footnote{This is currently broken -- each player
is automatically assigned a first legion marker even if ``Auto pick markers''
is off.}

After each player has picked his initial legion marker, the MasterBoard window
will pop up. You'll see each player's initial legion marker sitting in a tower 
hex. You'll also see a small window in the lower right corner of the screen. 
This is the Game Status window, which tracks each player's score, number of 
legion markers remaining, etc. And there's a Caretaker window which tracks how 
many of each creature remain in the stacks. (These are both optional: turn them
on or off from the Graphics menu.)

You can right-click (or control-click if you have a one-button mouse) on a 
legion to see its contents. (Unless you've selected ``All stacks visible'' on
the Game menu, the contents of other players' legions are hidden, except for 
those creatures which have recently been revealed via fighting, recruiting, or
teleporting). You can right-click on a hex to call up a menu, which lets you 
either see what you can recruit in that hex, or its battle map.

The active player first needs to split his initial 8-high legion. You'll
notice that the hex containing the active player's legion is lit up as a
reminder; in future turns this will happen for all 7-high legions. (It's
also legal to split legions with 4-6 characters in them, even though they
are not highlighted.) Click on the legion. A dialog will come up to let
you pick the new legion marker to use. Then another dialog will come up,
allowing you to move characters between the two legions. The game will
not let you leave the split phase on the first turn until each of your
legions contains three Creatures and one Lord. When you're ready, select
Done from the Phase menu.

Next comes the movement phase. The game will tell you your movement roll.
Click on a legion, and the places it can move will light up. Little pictures of
the creatures the legion can recruit in hex will also appear. Click on one of 
those places, and the legion will move there. (In some cases, you will need to 
choose whether to teleport or move normally, or which lord teleports if there 
is more than once choice.) The ``Undo Last Move'' and ``Undo All Moves'' 
actions are there in case you change your mind. During the first turn only, 
and once only, there will be a ``Take Mulligan'' action which you can use to 
re-roll your movement. When you're done moving everything you want to move, 
select Done.\footnote{If you look closely at the menus you'll see that many of 
the frequently used options have hotkeys, like 'd' for "Done."}

If you moved any legions onto enemy legions, then next comes the engagement
phase. Each hex with an engagement will light up. Click on the one you want
to resolve first, and a window will pop up showing both legions and giving the
defender a chance to flee, if applicable. If the defender doesn't flee, then
the attacker is given a chance to concede. If both legions stick around, then
a negotiation window pops up\footnote{Negotiation is currently broken.
It'll be back in soon.}, where creatures can be clicked on to ``X'' them
away. If the combatants can come to an agreement where all the creatures on
at least one side die, then there's no need to fight. Otherwise, it's time
for battle.

During a battle, the appropriate BattleMap pops up, with each legion on the
appropriate entry side. (You have to choose an entry side during movement
when more than one is possible.) The defender goes first. Click on each
character, and the places it can move light up. Click on one of those
places, and the character moves there. Repeat until all characters are
on-board, unless you'd like to leave some off-board to die for some reason.
The ``Undo Last Move'' and ``Undo All Moves'' menu options are available. When
done moving, click Done. The attacker repeats the process, except that 
after he finishes moving, it's striking time.

Any creatures adjacent to an enemy must strike; rangestrikers with an enemy in
range and line of sight may strike. (If you turn on the ``Auto forced strike''
option, then creatures that are forced to strike and have only one legal
target will strike first without any intervention on your part, which speeds
things up a bit.) Click the striker, and all his legal targets light up. 
Pick one, and he tries to strike it. (If it's legal to take a strike penalty 
in order to carry, then a dialog will pop up to ask if you want to do so.) 
The number of hits are displayed on the target. If the target is dead, it 
will have a big ``X'' displayed over it. If there is excess damage that can 
legally carry over, then the legal carry target(s) will light up, and the
cursor should change to a number, and the striking player needs to pick 
which one to carry to, or click somewhere else to decline the carry. This 
carry process can repeat if the strike blows through more than one creature. 
There's no way to undo strikes. (That would be cheating.) When done 
striking, choose Done. 

After the strike phase, the other player gets a strikeback phase. It's
identical to the strike phase, except that rangestrikes are not allowed.
Dead creatures do get to strike back before being removed.

The first turn after he kills an opposing character, the attacker may be
allowed to summon an angel or archangel, if there is one available in an
unengaged legion, and he hasn't yet summoned an angel this turn, and the
legion doesn't already have seven creatures. If so, a dialog will appear
and all MasterBoard hexes with summonable angels will light up. The
attacker must click on one of those hexes, then select the angel or
archangel as appropriate in the dialog.

During turn 4 of the battle, the defender may be allowed to muster a recruit.
If so, a dialog will pop up showing the legal recruits. If desired, pick one.
If no recruit is desired, dismiss the recruit dialog. (Click on the X in the
top right corner, or double-click the top left corner, depending on how you
normally dismiss dialogs in your OS.)

When the battle finishes, the winner gets some points and maybe the option of
acquiring one or more angels or archangels. If the winner didn't summon
an angel or recruit a reinforcement earlier, he will get another choice if
eligible.

After all engagements are resolved, choose Done to proceed to the mustering 
phase. Legions that moved and can recruit will light up. Click on each one 
and choose a recruit. If more than one type of creature is capable of 
summoning that recruit, you'll have to choose the recruiter(s) to be 
revealed, unless the ``Autopick recruiter'' option has been selected.
When done, click Done and pass the mouse to the next player.

The game ends when zero or one Titans remain. The last player standing is
the winner; if the game ends with a mutual elimination, it's a draw.

\section{Improvements}

If you find any bugs that you think we can fix, please let us know, in
as much detail as possible. (In particular, include the OS and JVM 
version.) The best way to report bugs is via the bug tracker at 
SourceForge -- go to ]
\htmladdnormallink{http://colossus.sf.net}{http://colossus.sf.net},
click on the SourceForge icon, and click on Bugs. (If that's too hard
you can just send email.)

We've tried to get the rules right, though a few areas (concession timing,
in particular) are still off. Bruno Wolff's Titan Errata and Clarifications at
\htmladdnormallink{http://wolff.to/titan/errata.shtml}{http://wolff.to/titan/errata.shtml}
is a good place to check for rules issues.

\section{Colossus and Titan related links}

\begin{itemize}

\item Bruno's Titan Home Page,
\htmladdnormallink{http://wolff.to/titan/}{http://wolff.to/titan/}.
You'll find there the
\htmladdnormallink{Titan Errata and Clarifications}{http://wolff.to/titan/errata.html}

\item Jerry's Titan page,
\htmladdnormallink{http://www.coiinc.com/people/guerrero/}{http://www.coiinc.com/people/guerrero/}.

\item Chee-Wai's Titan page,
\htmladdnormallink{http://www.cheewai.com/titan/}{http://www.cheewai.com/titan/}.

\end{itemize}

\end{document}
